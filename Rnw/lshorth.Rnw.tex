%%!TEX root = ../tex/lshorth.tex
%$Header: /u/math/j40/cvsroot/lectures/src/Rstuff/lshorth/Rnw/lshorth.Rnw.tex,v 1.5 2007/08/12 11:56:08 j40 Exp $
% -*- mode: noweb; noweb-default-code-mode: R-mode; -*-
%
%  Adjust the path names below and run the R command to process this file
%
% lshorth implementation
%
% oldoptions<-options(width=72,prompt="  ", continue="  "); setwd("/users/gs/documents/lectures/src/rstuff/lshorth/rnw_out/");Sweave(file= "../rnw/lshorth.rnw.tex", output="lshorth.tex",debug=TRUE,eps=FALSE);options(oldoptions)
%
% package.skeleton(name="lshorth",list=c("lshorth","plot.lshorth"))
%
%* Run R CMD check to check the package tarball.
%* Run R CMD build to build the package tarball.
% \usepackage{Sweave} %Don' use it. We define our own styles.
%:R: shorthp

This is the implementation base for \ircode{lshorth}.
The technical note is kept in a separate document.
\section{lshorth}
First catch the special case $p \leq \frac{1}{count}$ where $lenvec_i=0$. For other cases the $p$-shorth interval has the form $[x_{(j)}, x_{(j + \Delta)}]$. We build 
$$
lenvec_j = x_{(j+\Delta  )} -   x_{(j)}
$$

Function \irfunx{lshorth} returns the length of $p$-shorth as a matrix in a structure element \ircode{lshorth} . Entry [p,i] gives the length of the $psteps_p$ shorth containing $x_{(i)}$.
<<>>=
options(keep.source=TRUE)
lshorth<- function(x,probs=NULL, plot=TRUE,...){
    if (!is.numeric(x))
	stop("'x' must be numeric")
	
    if (length(x)<1)
	stop("'x' must have positive length")
	count <- length(x)	
	# if (is.null(psteps)) {psteps <- 2*ceiling(log2(length(x))+1)}
	
		if (is.null(probs)){
			ppx <-ceiling(log2(count) / 2) 
			probs <- c( 1/2^(ppx:1),1-1/2^(2:ppx))
			ppx<-NULL
			}
	# if (is.null(probs)) {probs <- (1:psteps)/(psteps + 1)}
    
    xname <- paste(deparse(substitute(x), 60), collapse="\n")
    xsort <- sort(x)
	
	shorthm<-matrix(ncol=length(probs),nrow=length(x))
	for (px in 1:length(probs)){
		#cat("shorthm ",px,probs[px],"\n")
	#shorthm[,px]<- shorthp(x,probs[px])
	{
    #cat("Enter shorthp","\n")
    count <- length(x)
    # Catch special case $probs \leq \frac{1}{count}
    if (probs[px]<=1/count)  {
        shorthvec <- rep.int(0, count)
    }
    else {
    #The $p$-shorth interval has the form $[j, j + \Delta]$. We build 
    #lenvec_j = x_{(j+\Delta :n )} -   x_{(j:n)}
    jDelta <- ceiling(probs[px] * count) - 1
    jmax <- count - jDelta
    lenvec <- xsort[-(1:(jDelta))] - xsort[-((count - jDelta + 1):count)]
    # cat("lenvec\n");print(lenvec)
   
    shorthvec <-vector(mode="numeric",count)
    # special case i==1. This is the start of the iteration
    minlenj <- 1
    shorthvec[1] <- minlen <- lenvec[1]
     #cat(i,jmin,jmax,minlenj,shorthvec[i],"\n")
    # general case
    for (i in 2:count) {
        jmin <- i - jDelta
        if (jmin < 1) {
            jmin <- 1
        }
        if (i + jDelta > count) {
             minlenj <- jmin - 1 + which.min(lenvec[jmin:(count - jDelta)])
        }
        else {
           #cat("nice cases \n")
            if (minlenj < jmin) {
               #cat("old min too small \n")
                minlenj <- jmin - 1 + which.min(lenvec[jmin:i])
            }
            else {
               #cat("old min in range. default:keep\n")
                if (lenvec[i] < minlen) {
                  minlenj <- i
                 }
            }
        }
        shorthvec[i] <- minlen  <- lenvec[minlenj]
        #cat(i,jmin,minlenj,shorthvec[i],"\n")
    } #for i
    } # if (probs<=1/count) else
    #return(shorthvec)
    shorthm[,px]<-shorthvec
} #shorthp

	}
	shorthm <- structure(list(x=xsort, xname=xname, lshorth = shorthm, probs=probs), class="lshorth")
	if (plot) {plot(shorthm, probs=probs, xlab=xname,...)}
	invisible(shorthm)
	}
@
\section{plot.lshorth}
The plotting function may be improved. 

\begin{itemize}
\item So far, the shorth length axis points downward: small shorth length corresponds to high density. With this orientation, the shorth plot can be compared to a density plot.

\item The plot is currently scaled to the empirical shorth range.

\item A visual feedback for the boundary effects should be given, e.g. trim shorth where $x_{(i)}$ is too near to the boundary of the $x$ range.

\item Clean up calling convention. Make reusable.
 
\end{itemize}

%:R:plot.lshorth
<<>>=

plot.lshorth <- function(x, y, 
		probs=(1:psteps)/(psteps+1), 
		psteps=9,
		main="Shorth", xlab=NULL, 
		ylab=NULL,legend=TRUE, ...)
{
	lshorthx <- x #fix me
	shorthm <- t(lshorthx$lshorth)
	if (is.null(xlab)) {
        xlab <- x$xname
    }
	if (is.null(ylab)) {
		ylab <- "std lshorth"
	}
	
	shorthl <-min (lshorth(x=lshorthx$x,0.5,plot=FALSE)$lshorth)# the length of the shorth
	shorthmy <- shorthm/shorthl
	shorthmy <- (max(shorthmy)-shorthmy)
	
	ylim <- c(0,max(shorthmy)*1.1)
	if (is.na(ylim[2])) {
		ylim[2]<-0
	}
	
	plot.new()
	plot.window(xlim=range(lshorthx$x[is.finite(lshorthx$x)]),ylim=ylim, ...)
	axis(1)
	axis(2)
	title(main=main, xlab=xlab, ylab=ylab,)
	rug(lshorthx$x)
	
	lwd <- ceiling(6*(0.5-abs(probs-0.5)))
	
	for (px in 1:length(probs)){
	#	lwd <- ceiling(6*(0.5-abs(probs[px]-0.5)))
		lines(lshorthx$x,shorthmy[px,],lwd=lwd[px])
	}
	if (legend){
	temp <- legend("topright", legend = rep(" ",length(probs)),
               text.width = strwidth("0.000"),
               lwd = lwd, xjust = 1, yjust = 1,
               title = "Coverage")
	text(temp$rect$left + temp$rect$w, temp$text$y,
     format(probs,digits=3), pos=2)
	}
	invisible(shorthm)
}

@
\ifx\private\undefined%
\else
\noindent
{\tiny%
\verb+$Source: /u/math/j40/cvsroot/lectures/src/Rstuff/lshorth/Rnw/lshorth.Rnw.tex,v $+\\
\verb+$Revision$+\\
\verb+$Date$+\\
\verb+$Name:  $+\\
\verb+$Author$+\\
\begin{verbatim}
$Log: lshorth.Rnw.tex,v $
Revision 1.5  2007/08/12 11:56:08  j40
ylim  in plot modified

\end{verbatim}
}
\fi
\endinput

